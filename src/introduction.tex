\chapter{Introduction}
\section{Positron Emission Tomography}
Positron Emission Tomography (PET) is a functional imaging technique widely used in clinical and research settings to monitor physiological processes.
In PET, a biologically active molecule is labeled with a positron-emitting radioisotope, known as radiotracer and then injected into the body.
As the radiotracer accumulates in target tissues, its radioactive decay produces positrons, which interact with electrons to emit pairs of gamma photons in nearly opposite directions.
These photons are detected by the PET scanner, and through image reconstruction algorithms, a three-dimensional representation of tracer distribution is generated.
This imaging modality allows for the investigation of metabolic changes, receptor binding, and other biochemical processes, providing invaluable information in oncology, neurology, cardiology, and other fields.

There are two main methods of acquiring PETs.
Dynamic and static imaging.
Static PET involves acquiring a single scan after the radiotracer injection.
This single snapshot offers a powerful yet simplified view of tracer distribution.
The common quantification metric in static imaging is the Standardized Uptake Value (SUV).
The SUV normalizes tissue uptake by injected dose and patient weight, allowing a semi-quantitative comparison of tracer accumulation across different tissues or over time. \cite{TODO}
Due to its simplicity, static PET is widely used in clinical settings but it also comes with limitations.
Because it reflects just one time point, the SUV cannot capture the temporal dynamics of how a tracer is taken up and cleared, and various physiological factors can influence its measurements reducing its accuracy.


Dynamic PET provides a more comprehensive view of the radiotracer kinetics by acquiring a series of images over a period of time-ranging from few minutes to more than an hour depending on the tracer type-immediately following tracer administration.
Instead of a single snapshot, dynamic imaging produces time-activity curves (TAC), presenting how tracer concentration in each tissue changes over the entire scanning period.
This allows for measuring physiological parameters such as tracer rate of influx (\(K_i\)), volume of distribution (\(V_T\)), and rates of phosphorylation and dephosphorylation.
This greater depth of information is especially valuable in research settings where... \pdfcomment{i'm not sure?}.

% Dynamic PET, however, places a higher burden on both the patient and the imaging facility.
% It requires longer scanning sessions, more complex data analysis, and can be logistically demanding, which is why it remains more common in research or specialized clinical scenarios rather than high-throughput diagnostic workflows.

\section{Compartmental Model}
To measure these parameters of interest, kinetic modeling must be carried out. Number of graphical models (e.g. Logan \cite{TODO} and Patlak \cite{TODO}) have been proposed but compartmental modeling is the most popular and is considered as the most accurate of kinetic modeling.
Compartment modeling is a mathematical framework where the distribution and kinetics of a radiotracer are described by dividing the system into distinct “compartments,” each representing pools of tracer in blood or organs that behave in a uniform or homogeneous way and interact with other compartments.
Interaction can be one way or two way with other compartments, meaning the tracer can either enter and leave or just enter the compartment.

Here, we will focus on two-tissue compartment model (2TCM) or also known as three compartment model in series mode with FDG as the tracer.
It consists of one compartment or one tissue for the free tracer \(C_F(t)\) and another for the receptor-bounded tracer \(C_B(t)\), plus an external compartment that represents the concentration of the tracer in the plasma or the blood which is known as the input function \(C_P(t)\).

The tracer kinetics is governed by a series of first-order differntial equations where the exchange rate between the compartments is considered as constants:

\begin{align}
	\frac{dC_F(t)}{dt} & = K_1 \, C_P(t) \;-\; \bigl(k_2 + k_3\bigr) C_F(t) \;+\; k_4 C_B(t) \,, \label{eq:2tcm-c1} \\[6pt]
	\frac{dC_B(t)}{dt} & = k_3 \, C_F(t) \;-\; k_4 \, C_B(t), \label{eq:2tcm-c2}
\end{align}
where  \(K_1, k_2, k_3, k_4\) are the constant rates.

It's considered that once FDG is phosphorylated (bounded), there is little to no dephosphorylation back to the free compartment (unbounding).
Meaning, we can can consider \(k_4=0\) \cite{TODO}.
This is why this variant is called the irreversible two-tissue compartment model.
Thus Equation~\eqref{eq:2tcm-c1} and \eqref{eq:2tcm-c2} simplifiy to

\begin{align}
	\frac{dC_F(t)}{dt} & = K_1 \, C_P(t) \;-\; \bigl(k_2 + k_3\bigr) C_F(t)  \,, \\[6pt]
	\frac{dC_B(t)}{dt} & = k_3 \, C_F(t) ,
\end{align}

The total tissue and blood concentration \(C_T(t)\) that is measured in PET (i.e. the PET signal) is the
\begin{equation}
	C_T(t) \;=\; C_F(t) \;+\; C_B(t) \;+\; C_P(t).
\end{equation}

The parameters \(K_1, k_2, k_3\) can be estimated by fitting the model to the measured PET TACs, using the input function as \(C_P(t)\).
At the end, we can derive \(K_{i}\) or the influx rate (trapping rate) of FDG in the tissue as
\begin{equation}
	K_i = \frac{K_1 \times k_3}{k_2 + k_3}
\end{equation}

\begin{figure}
	\centering
	\begin{tikzpicture}[>=latex, thick]
		\node[draw, rounded corners, minimum width=2.5cm, minimum height=2cm, align=center] (Cp) at (0,0) {$C_P$};
		\node[draw, rounded corners, minimum width=2.5cm, minimum height=2cm, align=center] (C1) at (5,0) {$C_F$};
		\node[draw, rounded corners, minimum width=2.5cm, minimum height=2cm, align=center] (C2) at (10,0) {$C_B$};

		\draw[->]
		([yshift=8pt]Cp.east) to[out=0, in=180]
		node[above] {\(K_1\)}
		([yshift=8pt]C1.west);

		\draw[->]
		([yshift=-8pt]C1.west) to[out=180, in=0]
		node[below] {\(k_2\)}
		([yshift=-8pt]Cp.east);

		\draw[->]
		([yshift=8pt]C1.east) to[out=0, in=180]
		node[above] {\(k_3\)}
		([yshift=8pt]C2.west);

		\draw[->, dashed, opacity=0.5]
		([yshift=-8pt]C2.west) to[out=180, in=0]
		node[below] {\(k_4 = 0\)}
		([yshift=-8pt]C1.east);

		\draw[dashed, rounded corners, thick] ($(C1.north west)+(-0.3,0.3)$) rectangle ($(C2.south east)+(0.3,-0.3)$);

	\end{tikzpicture}
	\caption{Schematic of the irreversible two-tissue compartment model (2TCM)}
	\label{fig:2tcm}
\end{figure}

FDG is an analog of glucose, not glucose itself. To convert the FDG trapping rate to the actual rate of glucose metabolism, the glucose concentration and the lumped constant is taken in to account.
\begin{equation}
	\textrm{MR}_{glu} (\textrm{~\textmu mol/min/100g}) = \frac{[C]}{LC} \cdot K_i
\end{equation}
where \(\textrm{MR}_{glu}\) is the metabolic rate of glucose, \([C]\) is the glucose concentration, and \(LC\) is the lumped constant. \pdfcomment{what are [c] and LC units?}

\section{Input Function}
\subsection{Arterial Input Function}
Arterial input function (AIF) is considered the gold standard measure of deriving the input function.
This is done by inserting an arterial catheter in to the patient and continously drawing blood sample and measuring its radiotracer concentration to obtain the blood TAC to be used in the quantification model.
This procedure is invasive and causes discomfort for the patient which could subsequently discourage them from participating in future examinations.
Moreover, this method is labour intensive and requires trained personnel to handle the patient and the measuring devices.

\subsection{Image Derived Input Function}
Image derived input function (IDIF) is proposed as an alternative non-invasive method of obtaining the input function.
IDIF techniques typically involve identifying vascular structures or regions with high blood pool activity within the imaging field and extracting the input function from the PET image.
Carotids are the largest arteries included in the limited filed of view (FOV) of brain PET with a diameter of about 5 mm which is comparable to a typical PET spatial resolution.
So the extraction of the carotids directly from the PET would be impractical due the limited resolution of PET images and strong partial volume effect (PVE) present.

With the emergence of hybrid PET/MRI machines, simultaneous acquisition functional and anatomical data has became possible, with MRI providing high-resolution soft tissue contrast and PET capturing metabolic activity.
Time-of-flight (TOF) MR angiography (MRA) provides high contrast in arteries which can be used for accurate segmentation of the carotids.
Even with an accurate segmentation of the arteries, the input function derived by directly applying the mask on to the PET image would suffers from partial volume effects which need correction.


\citeauthor{irace2021bayesian} \cite{irace2021bayesian} proposed method for IDIF estimation by an MRA-driven carotids segmentation and a bayesian framework for incorporating prior knowledge in to a geometric transfer matrix method -a classical partial volume correction method. The aim here was to improve on this method and increase accuracy.
